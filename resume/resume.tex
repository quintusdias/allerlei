\documentclass[11pt]{moderncv}  % Sets the default text size to 11pt and class
                               % to article.
% moderncv themes
% style options are 'casual' (default), 'classic', 'oldstyle' and 'banking'
\moderncvstyle{casual}

% color options 'blue' (default), 'orange', 'green', 'red', 'purple',
% 'grey' and 'black'
\moderncvcolor{blue}

% to set the default font; use '\sfdefault' for the default sans serif font,
%   '\rmdefault' for the default roman one, or any tex font name
%\renewcommand{\familydefault}{\sfdefault}

% uncomment to suppress automatic page numbering for CVs longer than one page
%\nopagenumbers{}

% character encoding
\usepackage[utf8]{inputenc}

% adjust the page margins
\usepackage[scale=0.75]{geometry}

% if you want to change the width of the column with the dates
%\setlength{\hintscolumnwidth}{3cm}

% for the 'classic' style, if you want to force the width allocated to your
%   name and avoid line breaks. be careful though, the length is normally
%   calculated to avoid any overlap with your personal info; use this at your
%   own typographical risks...
%\setlength{\makecvtitlenamewidth}{10cm}

% personal data
\name{John}{Evans}

%\title{Resumé title}
% optional, remove / comment the line if not wanted

\address{1743 18th Street NW}{Washington, DC  20009}{USA}
\phone[mobile]{+1~(617)~447~0030}                   
\email{john.g.evans.ne@gmail.com}                  
\social[twitter]{johnevans007}                    
\social[github]{quintusdias}                     
%\extrainfo{additional information}             

%\photo[64pt][0.4pt]{picture}
% optional, remove / comment the line if not wanted.
% '64pt' is the height the picture must be resized to,
% 0.4pt is the thickness of the frame around it (put it to 0pt for no frame)
% and 'picture' is the name of the picture file

\hyphenation{MATLAB}

%-------------------------------------------------------------------------------
%            content
%-------------------------------------------------------------------------------
%            content
%-------------------------------------------------------------------------------
\begin{document}
\makecvtitle

\renewcommand{\cvcomputer}[2]{\cvline{#1}{\small#2}}
\section{Computer skills}
\cvcomputer{Languages}
{
    Strong in Python (numpy, pandas, matplotlib, unittest, psycopg2).
    Familiar with ArcGIS and PostgreSQL server environments.  Several
    years experience with MATLAB, C/C++, Bash.  Past experience with R,
    jQuery, SAS, \LaTeX, Java, PHP, Perl, Fortran 77/90, 
    database administration with MySQL.  Some
    experience with Oracle.
}
\cvcomputer{Software Practices}
{
    Experience with continuous integration, writing test plans,
    unit testing, and software version control packages such as
    Git, Subversion, Perforce, and CVS.
}
\cvcomputer{File Formats}
{
    Extensive experience with netCDF, HDF5, HDF4, JPEG2000, CFITSIO,
    libTIFF, CDF, DICOM, NRRD, and NIfTI.  Some experience with
    OpenJPEG, Kakadu, and IJG JPEG libraries.
}
\cvcomputer{Platforms and Systems \mbox{Administration}}
{
	Currently working with RHEL 7, Fedora, and Mac OS X.  Past
	experience configuring Mandriva, Scyld, Ubuntu, Digital
	Unix, Irix, Windows.  Past experience configuring Samba,
	Amanda, Apache, iptables, FLEXLM.
}

\section{Open Source Projects}
\cvlistitem
{
    \href{http://glymur.readthedocs.org}{\texttt{Glymur}}:  Tools for accessing
	JPEG 2000 files.
}
\cvlistitem
{
    \href{https://github.com/hdfeos/zoo_python}{\texttt{HDF-EOS Python Zoo}}:
	Examples codes in Python for HDF-EOS.
}
\cvlistitem
{
    \href{http://python-xmp-toolkit.readthedocs.org}
	{\texttt{Python XMP Toolkit}}: a library for reading and writing XMP
	metadata.
}
\cvlistitem{
    \href{http://mexcdf.sourceforge.net}{\texttt{Mexcdf}}:
}
\cvlistitem
{
    \href{http://www.sunpy.org}{\texttt{SunPy}}
}

\section{Experience}
\cventry
{2015 -- 2018}
{Software Engineer}
{UCAR}{Washington, DC}{}
{
	24/7 operational support and monitoring of 
	\href{https://nowcoast.noaa.gov}{\texttt{nowCOAST}}
	and
	\href{https://idpgis.ncep.noaa.gov}{\texttt{IDP-GIS}}.  System
	administration of ArcGIS server farms.  Daily participation
	at NCEP Central Operations "OPS" meetings to review prior
	24 hours operations.  Created ISO 19115-2 metadata records
	with custom Python scripts for over 200 web mapping services.
}
\cventry
{2014 -- 2015}
{Programmer}
{Integrated Statistics/NOAA}{Woods Hole, MA}{}
{
    Database programming support for recreational fisheries assessments
    at the Northeast Fisheries Science Center. Wrote Django front-end
    for interactive exploration of scallops population model.
}
\cventry{2013 -- 2014}{Programmer}
{
	Partners Health Care / Bullfinch Medical Group}{Boston, MA}{}{
    Rewrote software pipelines for processing diffusion and DCE medical
    imaging data.  Performed exploratory work on platforms such as
    HubZero  for comparing algorithm performance. Wrote documentation
    and analysis tools for disease propagation model.
}
\cventry{2007 -- 2012}{Developer}{MathWorks}{Natick, MA}{}
{   
    Team lead for Image and Scientific Data Formats.  Delivered
    new MATLAB functionality for netCDF, HDF5, JPEG2000, HDF,
    HDF-EOS, FITS, TIFF, and CDF formats, including both high-level
    and low-level interfaces.  Also responsible for day to day
    maintenance of existing support for JPEG, PNG, GIF and various
    other image formats.  Delivered netCDF formal interface feature
    into MATLAB R2008b release, updated for netCDF-4 support in
    R2010b. Triaged bug reports, improved code coverage and quality,
    and managed 3rd party library integration and general infrastructure
    of my team's particular software component.  Wrote and presented
    6-month team planning documents to company management.  Sponsored
    five short-term intra-company internship projects for new
    engineers.  Represented MathWorks at annual HDF/NASA Spring
    Workshops.
}
\cventry{2004--2007}{Computing Specialist}
{
	Institute of Coastal and Marine Science, Rutgers University}
    {New Brunswick, NJ}{} {
    Assisted in setup,  operation, and output analysis  of Regional
    Ocean Model System (ROMS) on various OpenMP and MPI systems.
    This included operational support of a Beowulf Cluster with 20
    dual processor compute nodes, system administration for a number
    of Linux workstations and laptops, administration of a tomcat
    server.
}
\cventry{2003--2004}{Computing Specialist}
{
	USGS Branch of Atlantic Marine Geology}{Woods Hole, MA}{}{
    Conversion of aging geographic datasets into ESRI grid and shape
    file formats and Created GIS of Gulf of Mexico datasets.
}
\cventry{2001--2003}
{Computing Specialist}
{School of Marine Science, University of Maine}
{Orono, ME}{}
{
    Designed and implemented a near real-time buoy processing system
    for data telemetered via both cell phone and GOES satellite for
    GoMOOS (Gulf of Maine Ocean Observing System).   The array of
    instruments included wind sensors, ADCPs, current meters,
    conductivity/temperature recorders, accelerometers, radiometers,
    and fluorometers.   Responsible for MySQL database administration
    and general systems administration of a small network of RedHat
    Linux workstations, Windows PCs, and Macs.   Wrote documentation
    for the processing system and system administrative efforts.
    Also participated in CODAR project for assimilation of RADAR
    into surface current patterns.
}
\cventry
{1998--2000}
{Software Engineer}
{AER, Inc}
{Cambridge, MA}
{}
{
    Supported the Land Cover, Albedo, NDVI/EVI, and Imagery
    Environmental Data Records (EDRs) within the VIIRS project.
    Integrated new algorithms with legacy codes. Wrote gridding
    routines for sensed data to various common map projections.
}
\cventry{1995--1998}{Computing Specialist}
    {USGS Branch of Atlantic Marine Geology}{Woods Hole, MA}{}{
    Wrote support software for scientific visualization, modeling,
    and public education. Conversion of old time series data sets
    into standardized netCDF conventions.  Systems administration
    of Linux, Windows, Digital Unix, and Irix systems.}

\section{Education}
\cventry{1992--1995}
    {Master of Science}
    {Montana State University}
    {Bozeman}{}{Computer Science}
\cventry{1988--1990}
    {Master of Science}
    {Montana State University}
    {Bozeman}{}{Mathematics}
\cventry{1984--1998}
    {Bachelor of Science}
    {Montana State University}
    {Bozeman}{}{Applied Mathematics}

\section{Languages}
\cvitem{German}{Intermediate}
\cvitem{French}{Beginner}

\section{Volunteer activities}
\cvitem{Zooniverse}
  {Participant in Andromeda, Planet Hunters projects.}
\cvitem{Running Clubs}
  {Assisting in maintenance of Falmouth Track Club and Raritan Valley
   Road Runners web sites.
   President of Somerville Road Runners, June 1998 -- March 1999.}
\cvitem{MSU Athletics}
  {Statistician for Montana State University men's and women's basketball
  teams.}
\cvitem{Museum of the Rockies}
  {Front desk, education, and exhibit construction duties.}

\clearpage
%%-----       letter       ----------------------------------------------------
%% recipient data
\recipient{Kavli Institute for Astrophysics and Space}
{Massachusetts Institute of Technology\\77 Massachusetts Ave \#37241\\Cambridge, MA  02139}
%\date{January 01, 1984}
\opening{Dear Sirs,}
\closing{Sincerely,}

% use an optional argument to use a string other than "Enclosure", or redefine
% \enclname
%\enclosure[Attached]{curriculum vit\ae{}}

\makelettertitle

I would like to place my name into consideration for the software
support engineer position at the Kavli Institute.  I have followed
the events and discoveries of the Kepler Space Telescope over the
last few years with great interest and would be thrilled to contribute
to TESS.

Python has become my language of choice over the last few years and
I use it in most of my GitHub projects, but I also have a long
history with MATLAB.  In 2007, I joined the Image and Scientific
Data Formats team at the MathWorks, becoming  team lead after one
year.  This give me a broad experience with a variety of open source
image and scientific data formats and APIs, ranging from older
“classic” formats such as JPEG, FITS, TIFF, and HDF4, to newer and
more complex formats such as JPEG2000, HDF5, and of course netCDF.
In fact, my first assignment at MathWorks was to implement official
netCDF support in MATLAB, and this was largely based upon the earlier
netCDF community project on SourceForge, MEXCDF, to which I was the
prime contributor.   I also authored the R2011a “high-level” HDF5
interface for MATLAB while continuing to refactor and performance-tune
the existing “low-level” HDF5 interface.

Working at the MathWorks was a good experience in software engineering
practices such as continuous integration, refactoring, requirements
gathering, functional design, design reviews including test plans,
and long-range planning as the team lead.   There were not too many
opportunities for conferences, but for six years I did attend the
NASA/HDF Spring Workshop as MathWorks’ representative, where I
became well-acquainted with the people of the HDF5 Group.

Before coming to the MathWorks, I was at the School for Marine
Sciences at Rutgers University where I was responsible for managing
a small cluster used for running physical oceanographic models
(ROMS).  This required managing  netCDF output files, providing
general software support, usually in MATLAB but also in shell
scripting, maintaining a THREDDS/OpENDAP server, and general Linux
system administration.

Thank you for your consideration.  Please do not hesitate to contact
me if you require any further information.

\makeletterclosing

\end{document}


%% end of file `template.tex'.
