\documentclass[11pt,a4paper,sans]{moderncv}        
% possible options include font size ('10pt', '11pt' and '12pt'),
% paper size ('a4paper', 'letterpaper', 'a5paper', 'legalpaper',
%   'executivepaper' and 'landscape')
% and font family ('sans' and 'roman')

% moderncv themes
% style options are 'casual' (default), 'classic', 'oldstyle' and 'banking'
\moderncvstyle{casual}

% color options 'blue' (default), 'orange', 'green', 'red', 'purple', 'grey' and 'black'
\moderncvcolor{blue}

% to set the default font; use '\sfdefault' for the default sans serif font,
%   '\rmdefault' for the default roman one, or any tex font name
%\renewcommand{\familydefault}{\sfdefault}

% uncomment to suppress automatic page numbering for CVs longer than one page
%\nopagenumbers{}

% character encoding
\usepackage[utf8]{inputenc}

% adjust the page margins
\usepackage[scale=0.75]{geometry}

% if you want to change the width of the column with the dates
%\setlength{\hintscolumnwidth}{3cm}

% for the 'classic' style, if you want to force the width allocated to your
%   name and avoid line breaks. be careful though, the length is normally
%   calculated to avoid any overlap with your personal info; use this at your
%   own typographical risks...
%\setlength{\makecvtitlenamewidth}{10cm}

% personal data
\name{John}{Evans}

%\title{Resumé title}
% optional, remove / comment the line if not wanted

\address{36 Symphony Road 1A}{Boston, MA  02115}{USA}
\phone[mobile]{+1~(617)~447~0030}                   
\email{john.g.evans.ne@gmail.com}                  
\social[twitter]{johnevans007}                    
\social[github]{quintusdias}                     
%\extrainfo{additional information}             

%\photo[64pt][0.4pt]{picture}
% optional, remove / comment the line if not wanted.
% '64pt' is the height the picture must be resized to,
% 0.4pt is the thickness of the frame around it (put it to 0pt for no frame)
% and 'picture' is the name of the picture file

%-------------------------------------------------------------------------------
%            content
%-------------------------------------------------------------------------------
\begin{document}
\makecvtitle

\renewcommand{\cvcomputer}[2]{\cvline{#1}{\small#2}}
\section{Computer skills}
  \cvcomputer{Programming}{Strong in Python (numpy, pandas, matplotlib),
    strong in MATLAB, C/C++.  Familiar with R, \LaTeX, XMP/XML.  Past
    experience with Java, PHP, bash, SQL, DocBook/SGML, Perl, Fortran 77,
    Fortran 90}
  \cvcomputer{Software Practices}{Experience with continuous integration, writing
    test plans, unit testing (Python and MATLAB), and software version control
    packages such as Git, Subversion, Perforce, and CVS.}
  \cvcomputer{File Formats and Software Packages}{Extensive experience with netCDF,
    HDF5, HDF4, JPEG2000, CFITSIO, libTIFF, CDF, DICOM, NRRD, and NIfTI.  Some
    experience with OpenJPEG, Kakadu, and IJG JPEG libraries.  Several years of
    MySQL database administration, some experience with Oracle, PostgreSQL, SQLite}
  \cvcomputer{Platforms and Systems \mbox{Administration}}{Currently working with Fedora, CentOS,
    and Mac OS X.  Past experience configuring Mandriva, Scyld, Ubuntu,
    Digital Unix, Irix, Windows.  Past experience configuring Samba, Amanda,
    Apache, rsync, iptables, FLEXLM.}

\section{Open Source Projects}
  \cvlistitem{Glymur:  \href{glymur.readthedocs.org}{\texttt{http://glymur.readthedocs.org}}}
  \cvlistitem{Mexcdf:  \href{mexcdf.sourceforge.net}{\texttt{http://mexcdf.sourceforge.net}}}
  \cvlistitem{Sunpy:  \href{www.sunpy.org}{\texttt{http://www.sunpy.org}}}
  \cvlistitem{Python XMP Toolkit:  \href{python-xmp-toolkit.readthedocs.org}{\texttt{http://python-xmp-toolkit.readthedocsorg}}}

\section{Education}
\cventry{1992--1995}{Master of Science}{Montana State University}{Bozeman}{}{Computer Science}
\cventry{1988--1990}{Master of Science}{Montana State University}{Bozeman}{}{Mathematics}
\cventry{1984--1998}{Bachelor of Science}{Montana State University}{Bozeman}{}{Applied Mathematics}

\section{Experience}
\cventry{2013 -- 2014}{Programmer}{Partners Health Care / Bullfinch Medical Group}{Boston, MA}{}{
    Rewrote software pipelines for processing diffusion and DCE medical
    imaging data.  Performed exploratory work on platforms such as
    HubZero  for comparing algorithm performance. Wrote documentation
    and analysis tools for disease propagation model.}
\cventry{2007 -- 2012}{Developer}{MathWorks}{Natick, MA}{}{
    Team lead for Image and Scientific Data Formats.  Delivered new MATLAB
    functionality for netCDF, HDF5, JPEG2000, HDF, HDF-EOS, FITS, TIFF,
    and CDF formats, including both high-level and low-level interfaces.
    Also responsible for day to day maintenance of existing support for
    JPEG, PNG, GIF and various other image formats.  Delivered netCDF
    formal interface feature into MATLAB R2008b release, updated for
    netCDF-4 support in R2010b. Triaged bug reports, improved code
    coverage and quality, and managed 3rd party library integration and
    general infrastructure of my team's particular software component.
    Wrote and presented 6-month team planning documents to company
    management.  Sponsored five short-term intra-company internship
    projects for new engineers.  Represented MathWorks at annual HDF/NASA
    Spring Workshops.}
  \cventry{2004--2007}{Computing Specialist}
    {Institute of Coastal and Marine Science, Rutgers University}{New Brunswick, NJ}{}
    {Assisted in setup,  operation, and output analysis  of Regional
    Ocean Model System (ROMS) on various OpenMP and MPI systems.
    This included operational support of a Beowulf Cluster with 20
    dual processor compute nodes, system administration for a number of
    Linux workstations and laptops, administration of a tomcat server.}
  \cventry{2003--2004}{Computing Specialist}{USGS Branch of Atlantic Marine Geology}{Woods Hole, MA}{}
    {Conversion of aging geographic datasets into ESRI grid and shape
    file formats and Created GIS of Gulf of Mexico datasets.}
  \cventry{2001--2003}{Computing Specialist}{School of Marine Science, University of Maine}{Orono, ME}{}{
    Designed and implemented a near real-time buoy processing
    system for data telemetered via both cell phone and GOES satellite
    for GoMOOS (Gulf of Maine Ocean Observing System).   The array
    of instruments included wind sensors, ADCPs, current meters,
    conductivity/temperature recorders, accelerometers, radiometers,
    and fluorometers.   Responsible for MySQL database administration
    and general systems administration of a small network of RedHat
    Linux workstations, Windows PCs, and Macs.   Wrote documentation
    for the processing system and system administrative efforts.
    Also participated in CODAR project for assimilation of RADAR into
    surface current patterns.}
  \cventry{1998--2000}{Software Engineer}{AER, Inc}{Cambridge, MA}{}{
    Supported the Land Cover, Albedo,
    NDVI/EVI, and Imagery Environmental Data Records (EDRs) within the
    VIIRS project. Integrated new algorithms with legacy codes. Wrote
    gridding routines for sensed data to various common map projections.}
  \cventry{1995--1998}{Computing Specialist}{USGS Branch of Atlantic Marine Geology}{Woods Hole, MA}{}{
    Wrote support software for scientific
    visualization, modeling, and public education. Conversion of old
    time series data sets into standardized netCDF conventions.  Systems
    administration of Linux, Windows, Digital Unix, and Irix systems.}

\section{Languages}
\cvitem{German}{Intermediate}
\cvitem{French}{Beginner}

\section{Volunteer activities}
\cvitem{Zooniverse}{Participant in Andromeda, Planet Hunters projects.}
\cvitem{Running Clubs}{Assisting in maintenance of Falmouth Track Club and
    Raritan Valley Road Runners web sites.  Volunteer and MIT track meets.
    President of Somerville Road Runners, June 1998 -- March 1999.}
\cvitem{MSU Athletics}{Statistician for Montana State University men's and
    women's basketball teams.}
\cvitem{Museum of the Rockies}{Front desk, education, and exhibit construction duties.}

%\clearpage
%%-----       letter       ---------------------------------------------------------
%% recipient data
%\recipient{Company Recruitment team}{Company, Inc.\\123 somestreet\\some city}
%\date{January 01, 1984}
%\opening{Dear Sir or Madam,}
%\closing{Yours faithfully,}
%\enclosure[Attached]{curriculum vit\ae{}}          % use an optional argument to use a string other than "Enclosure", or redefine \enclname
%\makelettertitle
%
%Lorem ipsum dolor sit amet, consectetur adipiscing elit. Duis ullamcorper neque sit amet lectus facilisis sed luctus nisl iaculis. Vivamus at neque arcu, sed tempor quam. Curabitur pharetra tincidunt tincidunt. Morbi volutpat feugiat mauris, quis tempor neque vehicula volutpat. Duis tristique justo vel massa fermentum accumsan. Mauris ante elit, feugiat vestibulum tempor eget, eleifend ac ipsum. Donec scelerisque lobortis ipsum eu vestibulum. Pellentesque vel massa at felis accumsan rhoncus.
%
%Suspendisse commodo, massa eu congue tincidunt, elit mauris pellentesque orci, cursus tempor odio nisl euismod augue. Aliquam adipiscing nibh ut odio sodales et pulvinar tortor laoreet. Mauris a accumsan ligula. Class aptent taciti sociosqu ad litora torquent per conubia nostra, per inceptos himenaeos. Suspendisse vulputate sem vehicula ipsum varius nec tempus dui dapibus. Phasellus et est urna, ut auctor erat. Sed tincidunt odio id odio aliquam mattis. Donec sapien nulla, feugiat eget adipiscing sit amet, lacinia ut dolor. Phasellus tincidunt, leo a fringilla consectetur, felis diam aliquam urna, vitae aliquet lectus orci nec velit. Vivamus dapibus varius blandit.
%
%Duis sit amet magna ante, at sodales diam. Aenean consectetur porta risus et sagittis. Ut interdum, enim varius pellentesque tincidunt, magna libero sodales tortor, ut fermentum nunc metus a ante. Vivamus odio leo, tincidunt eu luctus ut, sollicitudin sit amet metus. Nunc sed orci lectus. Ut sodales magna sed velit volutpat sit amet pulvinar diam venenatis.
%
%Albert Einstein discovered that $e=mc^2$ in 1905.
%
%\[ e=\lim_{n \to \infty} \left(1+\frac{1}{n}\right)^n \]
%
%\makeletterclosing
%
%%\clearpage\end{CJK*}                              % if you are typesetting your resume in Chinese using CJK; the \clearpage is required for fancyhdr to work correctly with CJK, though it kills the page numbering by making \lastpage undefined
\end{document}


%% end of file `template.tex'.
