\documentclass[11pt]{moderncv}  % Sets the default text size to 11pt and class
                               % to article.
% moderncv themes
% style options are 'casual' (default), 'classic', 'oldstyle' and 'banking'
\moderncvstyle{casual}

% color options 'blue' (default), 'orange', 'green', 'red', 'purple',
% 'grey' and 'black'
\moderncvcolor{blue}

% to set the default font; use '\sfdefault' for the default sans serif font,
%   '\rmdefault' for the default roman one, or any tex font name
%\renewcommand{\familydefault}{\sfdefault}

% uncomment to suppress automatic page numbering for CVs longer than one page
%\nopagenumbers{}

% character encoding
\usepackage[utf8]{inputenc}

% adjust the page margins
\usepackage[scale=0.75]{geometry}

% if you want to change the width of the column with the dates
%\setlength{\hintscolumnwidth}{3cm}

% for the 'classic' style, if you want to force the width allocated to your
%   name and avoid line breaks. be careful though, the length is normally
%   calculated to avoid any overlap with your personal info; use this at your
%   own typographical risks...
%\setlength{\makecvtitlenamewidth}{10cm}
%% recipient data
\recipient
{Portfolio Strat}
{57 Hartswood Road, London, United Kingdom, W12 9NE}

% personal data
\name{John}{Evans}

%\title{Resumé title}
% optional, remove / comment the line if not wanted

\address{1735 S Street NW}{Washington, DC  20009}{USA}
\phone[mobile]{+1~(617)~447~0030}                   
\email{john.g.evans.ne@gmail.com}                  
\social[twitter]{johnevans007}                    
\social[github]{quintusdias}                     
%\extrainfo{additional information}             

%\photo[64pt][0.4pt]{picture}
% optional, remove / comment the line if not wanted.
% '64pt' is the height the picture must be resized to,
% 0.4pt is the thickness of the frame around it (put it to 0pt for no frame)
% and 'picture' is the name of the picture file

\hyphenation{MATLAB}

%-------------------------------------------------------------------------------
%            content
%-------------------------------------------------------------------------------
%            content
%-------------------------------------------------------------------------------
\begin{document}

\makecvtitle

%%-----       letter       ----------------------------------------------------
\opening{Dear sirs,}
\closing{Sincerely,}

% use an optional argument to use a string other than "Enclosure", or redefine
% \enclname
%\enclosure[Attached]{curriculum vit\ae{}}

\makelettertitle

I would like to place my name into consideration for the Backend Python Engineer position.  Although most of my work experience has been more on the IT side of science institutions, I am a quick learner, am interested in branching out into the financial world, and feel I have a great deal to offer.

In the past few years, Python has become my language of choice, and it was my primary tool working in GIS Dissemination at the National Weather Service.  This involved such tasks as analyzing Apache log files, writing monitoring scripts, running JMeter load tests, and more.  Pandas and psycopg2 were probably the two packages that I relied upon the most, pandas for wrangling data and psycopg2 for realtime monitoring of the database underpinnings of the GIS web mapping services for which I comprised the 24/7 operational support.

Since leaving NWS, I've recently been trying to bolster my data science / machine learning credentials through online courses at EdX, which has emphasized Python and Spark.  I also currently maintain a GitHub project called Glymur that provides a Python interface to the OpenJPEG library.  Glymur has been included in the Arch, Debian, Gentoo, and Ubuntu distributions of Linux, and MacPorts as well.  Working on Glymur allowed me to learn a great deal about other Python packages such as matplotlib, scikit-image, and maybe most important of all, the unittest module.  Learning how to write good tests with unittest has been an ongoing project, and I am constantly surprised at learning some new (or old) feature and wondering how I ever got along without it.

Before working for NOAA/NWS in various capacities, I was a developer at the MathWorks, where I authored a number of features as a member and eventual team lead of the Image and Scientific Data Formats team.  Working at the MathWorks was a good experience in software engineering practices such as continuous integration, refactoring, requirements gathering, functional design, design reviews including test plans, and long-range planning.

Thank you for your consideration.  Please do not hesitate to contact me if you require any further information.

\makeletterclosing
\end{document}


%% end of file `template.tex'.
