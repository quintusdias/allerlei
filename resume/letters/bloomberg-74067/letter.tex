\documentclass[11pt]{moderncv}  % Sets the default text size to 11pt and class
                               % to article.
% moderncv themes
% style options are 'casual' (default), 'classic', 'oldstyle' and 'banking'
\moderncvstyle{casual}

% color options 'blue' (default), 'orange', 'green', 'red', 'purple',
% 'grey' and 'black'
\moderncvcolor{blue}

% to set the default font; use '\sfdefault' for the default sans serif font,
%   '\rmdefault' for the default roman one, or any tex font name
%\renewcommand{\familydefault}{\sfdefault}

% uncomment to suppress automatic page numbering for CVs longer than one page
%\nopagenumbers{}

% character encoding
\usepackage[utf8]{inputenc}

% adjust the page margins
\usepackage[scale=0.75]{geometry}

% if you want to change the width of the column with the dates
%\setlength{\hintscolumnwidth}{3cm}

% for the 'classic' style, if you want to force the width allocated to your
%   name and avoid line breaks. be careful though, the length is normally
%   calculated to avoid any overlap with your personal info; use this at your
%   own typographical risks...
%\setlength{\makecvtitlenamewidth}{10cm}
%% recipient data
\recipient
{Bloomberg HR}
{1101 New York Ave NW, Washington, DC 20005}

% personal data
\name{John}{Evans}

%\title{Resumé title}
% optional, remove / comment the line if not wanted

\address{1735 S Street NW}{Washington, DC  20009}{USA}
\phone[mobile]{+1~(617)~447~0030}                   
\email{john.g.evans.ne@gmail.com}                  
\social[twitter]{johnevans007}                    
\social[github]{quintusdias}                     
%\extrainfo{additional information}             

%\photo[64pt][0.4pt]{picture}
% optional, remove / comment the line if not wanted.
% '64pt' is the height the picture must be resized to,
% 0.4pt is the thickness of the frame around it (put it to 0pt for no frame)
% and 'picture' is the name of the picture file

\hyphenation{MATLAB}

%-------------------------------------------------------------------------------
%            content
%-------------------------------------------------------------------------------
%            content
%-------------------------------------------------------------------------------
\begin{document}
%%-----       letter       ----------------------------------------------------
\opening{Dear sirs,}
\closing{Sincerely,}

% use an optional argument to use a string other than "Enclosure", or redefine
% \enclname
%\enclosure[Attached]{curriculum vit\ae{}}

\makelettertitle

I would like to place my name into consideration for the Backend Engineer for Structured Products (Req 74076) position.
Although most of my work experience has been more on the IT
side of science, I am a quick learner and feel I have a great deal to offer.

Python became my language of choice over the last few years and was
my primary tool working in GIS Dissemination at the National Weather
Service.  This involved such tasks as analyzing Apache log files,
writing monitoring scripts, running JMeter load tests, and more.
Monitoring 24/7 was the primary focus of the position, however.
I primarily worked on the nowCOAST project, which in the month of
January 2019 recorded nearly 371 million hits and delivered over 14 TB of 
weather data.

I also have a long history with MATLAB.  In 2007, I joined the Image and
Scientific Data Formats team at the MathWorks, becoming  team lead after
one year.  Working at the MathWorks was a good experience in software
engineering practices such as automated testing (unfortunately with
in-house-only tools) and continuous integration, requirements gathering,
functional design, design reviews including test plans, code reviews,
and long-range planning as the team lead.   But since the MathWorks, I
used Python's unittest module in conjunction with Travis-CI for automated
testing of glymur, my JPEG2000 project on GitHub.

Working at the MathWorks gave me a broad experience with a variety
of open source image and scientific data formats and APIs, ranging
from older “classic” formats such as JPEG, FITS, TIFF, and HDF4,
to newer and more complex formats such as JPEG2000 and HDF5/netCDF.
In fact, my first assignment at MathWorks was to implement official
netCDF support in MATLAB, and this was largely based upon the earlier
netCDF community project on SourceForge, MEXCDF, to which I was the
prime contributor.   

Before coming to the MathWorks, I was at the School for Marine
Sciences at Rutgers University where I was responsible for managing
a small cluster used for running physical oceanographic models
(ROMS).  This required managing  netCDF output files, providing
general software support, usually in MATLAB but also in shell
scripting, maintaining a THREDDS/OpENDAP server, and general Linux
system administration.

Thank you for your consideration.  Please do not hesitate to contact
me if you require any further information.

\makeletterclosing

\end{document}


%% end of file `template.tex'.
