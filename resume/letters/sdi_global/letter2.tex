\documentclass[11pt]{moderncv}  % Sets the default text size to 11pt and class
                               % to article.
% moderncv themes
% style options are 'casual' (default), 'classic', 'oldstyle' and 'banking'
\moderncvstyle{casual}

% color options 'blue' (default), 'orange', 'green', 'red', 'purple',
% 'grey' and 'black'
\moderncvcolor{blue}

% to set the default font; use '\sfdefault' for the default sans serif font,
%   '\rmdefault' for the default roman one, or any tex font name
%\renewcommand{\familydefault}{\sfdefault}

% uncomment to suppress automatic page numbering for CVs longer than one page
%\nopagenumbers{}

% character encoding
\usepackage[utf8]{inputenc}

% adjust the page margins
\usepackage[scale=0.75]{geometry}

\hyphenation{MATLAB}

\begin{document}
\makecvtitle

\renewcommand{\cvcomputer}[2]{\cvline{#1}{\small#2}}
\section{Objective}
\cvitem{}
{    
    Secure a position in DevOps or development utilizing Python.
}

\section{Computer skills}
\cvcomputer{Certificates}
{
    {\textbf{DSE220x: Machine Learning Fundamentals}}:
    a course of study offered
    by UCSanDiegoX, an online learning initiative of UC San Diego
    through edX,
    \href{https://courses.edx.org/certificates/48ace65383f8411cbaffd0e7eb1d6447}{\texttt{https://courses.edx.org/certificates/48ace65383f8411cbaffd0e7eb1d6447}}
}

\cvcomputer{Languages and Platforms}
{
    Python - experience with numpy, pandas, matplotlib, seaborn,
    scikit-learn, requests, unittest, psycopg2, lxml, requests.
    Familiar with ArcGIS and PostgreSQL server environments.  Several
    years experience with MATLAB, C/C++, Bash, and writing scripts
    for Big Brother.  Past experience with \LaTeX, Java, PHP, Perl,
    Fortran 77/90, database administration with MySQL.  Some
    experience with Oracle, jQuery, R, SAS, Geoserver.  Load testing
    experience with JMeter.
}
\cvcomputer{Software Practices}
{
    Experience with continuous integration, writing test plans,
    unit testing, and software version control packages such as
    Git, Subversion, Perforce, and CVS.
}
\cvcomputer{Linux Systems}
{
    Currently working operationally with RHEL 6 and 7.  Past
    experience with Debian, Mandriva, Scyld, Ubuntu, Digital Unix,
    Irix and Irix.  Past experience configuring Samba, Amanda,
    Apache, iptables, FLEXLM.
}
\cvcomputer{File Formats}
{
    Extensive experience with netCDF, HDF5, HDF4, JPEG2000, CFITSIO,
    libTIFF, CDF, PNG, DICOM, NRRD, and NIfTI.  Some experience with
    OpenJPEG, Kakadu, and IJG JPEG libraries.
}

\section{Open Source Projects}
\cvlistitem
{
    \href{http://glymur.readthedocs.org}{\texttt{Glymur}}:  Tools
    for accessing JPEG 2000 files.
}
\cvlistitem
{
    \href{https://github.com/hdfeos/zoo_python}{\texttt{HDF-EOS
    Python Zoo}}: Examples codes written in Python for accessing
    HDF-EOS files.
}
\cvlistitem
{
    \href{http://python-xmp-toolkit.readthedocs.org} {\texttt{Python
    XMP Toolkit}}: a package interfacing with the exempi library
    for reading and writing XMP metadata.
}
\cvlistitem{
    \href{http://mexcdf.sourceforge.net}{\texttt{Mexcdf}}:  MATLAB
    utilities for reading and writing netCDF files.
}

\section{Experience}
\cventry
{Aug 2018 -- Feb 2019}
{Senior Production Analyst}
{Cyberdata}{Washington, DC}{}
{
	24/7 operational support and monitoring of 
	applications within IDP (Integrated Dissemination Program) at
	NOAA, including 
	\href{https://nowcoast.noaa.gov}{\texttt{nowCOAST}}
	and
	\href{https://idpgis.ncep.noaa.gov}{\texttt{IDP-GIS}}.
	System administration of ArcGIS server farms.  Daily
	participation at NCEP Central Operations "OPS" meetings to
	review prior 24 hours operations.  Wrote tools for analysis
	of web logs, running load tests with JMeter, and Big Brother
	scripts for realtime monitoring of operations, including
	test suites.  
}
\cventry
{May 2015 -- Aug 2018}
{Software Engineer}
{UCAR}{Washington, DC}{}
{
	24/7 operational support and monitoring of 
	\href{https://nowcoast.noaa.gov}{\texttt{nowCOAST}}
	and
	\href{https://idpgis.ncep.noaa.gov}{\texttt{IDP-GIS}}.
	System administration of ArcGIS server farms.  Daily
	participation at NCEP Central Operations "OPS" meetings to
	review prior 24 hours operations.  Wrote tools for analysis
	of web logs, running load tests with JMeter, and Big Brother
	scripts for realtime monitoring of operations, including
	test suites.  Created ISO 19115-2 metadata records with
	custom Python scripts for over 60 web mapping services.
	This position was similar to the Cyberdata position, except that
	it was limited to nowCOAST and IDP-GIS.
}
\cventry
{2014 -- 2015}
{Programmer}
{Integrated Statistics/NOAA}{Woods Hole, MA}{}
{
    Database programming support for recreational fisheries assessments
    at the Northeast Fisheries Science Center. Wrote Django front-end
    for interactive exploration of scallops population model.
}
\cventry{2013 -- 2014}{Programmer}
{
	Partners Health Care / Bullfinch Medical Group}{Boston, MA}{}{
    Rewrote software pipelines for processing diffusion and DCE medical
    imaging data.  Performed exploratory work on platforms such as
    HubZero  for comparing algorithm performance. Wrote documentation
    and analysis tools for disease propagation model.
}
\cventry{2007 -- 2012}{Developer}{MathWorks}{Natick, MA}{}
{   
    Team lead for Image and Scientific Data Formats.  Delivered
    new MATLAB functionality for netCDF, HDF5, JPEG2000, HDF,
    HDF-EOS, FITS, TIFF, and CDF formats, including both high-level
    and low-level interfaces.  Also responsible for day to day
    maintenance of existing support for JPEG, PNG, GIF and various
    other image formats.  Delivered netCDF formal interface feature
    into MATLAB R2008b release, updated for netCDF-4 support in
    R2010b. Triaged bug reports, improved code coverage and quality,
    and managed 3rd party library integration and general infrastructure
    of my team's particular software component.  Wrote and presented
    6-month team planning documents to company management.  Sponsored
    five short-term intra-company internship projects for new
    engineers.  Represented MathWorks at annual HDF/NASA Spring
    Workshops.
}
\cventry{2004--2007}{Computing Specialist}
{
	Institute of Coastal and Marine Science, Rutgers University}
    {New Brunswick, NJ}{} {
    Assisted in setup,  operation, and output analysis  of Regional
    Ocean Model System (ROMS) on various OpenMP and MPI systems.
    This included operational support of a Beowulf Cluster with 20
    dual processor compute nodes, system administration for a number
    of Linux workstations and laptops, administration of a tomcat
    server.
}
\cventry{2003--2004}{Computing Specialist}
{
	USGS Branch of Atlantic Marine Geology}{Woods Hole, MA}{}{
    Conversion of aging geographic datasets into ESRI grid and shape
    file formats and Created GIS of Gulf of Mexico datasets.
}
\cventry{2001--2003}
{Computing Specialist}
{School of Marine Science, University of Maine}
{Orono, ME}{}
{
    Designed and implemented a near real-time buoy processing system
    for data telemetered via both cell phone and GOES satellite for
    GoMOOS (Gulf of Maine Ocean Observing System).   The array of
    instruments included wind sensors, ADCPs, current meters,
    conductivity/temperature recorders, accelerometers, radiometers,
    and fluorometers.   Responsible for MySQL database administration
    and general systems administration of a small network of RedHat
    Linux workstations, Windows PCs, and Macs.   Wrote documentation
    for the processing system and system administrative efforts.
    Also participated in CODAR project for assimilation of RADAR
    into surface current patterns.
}
\cventry
{1998--2000}
{Software Engineer}
{AER, Inc}
{Cambridge, MA}
{}
{
    Supported the Land Cover, Albedo, NDVI/EVI, and Imagery
    Environmental Data Records (EDRs) within the VIIRS project.
    Integrated new algorithms with legacy codes. Wrote gridding
    routines for sensed data to various common map projections.
}
\cventry{1995--1998}{Computing Specialist}
    {USGS Branch of Atlantic Marine Geology}{Woods Hole, MA}{}{
    Wrote support software for scientific visualization, modeling,
    and public education. Conversion of old time series data sets
    into standardized netCDF conventions.  Systems administration
    of Linux, Windows, Digital Unix, and Irix systems.}

\section{Education}
\cventry{1992--1995}
    {Master of Science}
    {Montana State University}
    {Bozeman}{}{Computer Science}
\cventry{1988--1990}
    {Master of Science}
    {Montana State University}
    {Bozeman}{}{Mathematics}
\cventry{1984--1998}
    {Bachelor of Science}
    {Montana State University}
    {Bozeman}{}{Applied Mathematics}

\section{Languages}
\cvitem{German}{Intermediate}
\cvitem{French}{Beginner}

\section{Volunteer activities}
\cvitem{Zooniverse}
  {Participant in Andromeda, Planet Hunters projects.}
\cvitem{Running Clubs}
  {Assisting in maintenance of Washington Running Club, Falmouth
  Track Club and Raritan Valley Road Runners web sites.
   President of Somerville Road Runners, June 1998 -- March 1999.}
\cvitem{MSU Athletics}
  {Statistician for Montana State University men's and women's basketball
  teams.}
\cvitem{Museum of the Rockies}
  {Front desk, education, and exhibit construction duties.}
% if you want to change the width of the column with the dates
%\setlength{\hintscolumnwidth}{3cm}

% for the 'classic' style, if you want to force the width allocated to your
%   name and avoid line breaks. be careful though, the length is normally
%   calculated to avoid any overlap with your personal info; use this at your
%   own typographical risks...
%\setlength{\makecvtitlenamewidth}{10cm}
%% recipient data
\recipient
{Data Engineering Group}
{620 Eighth Avenue, New York, NY 10018}

% personal data
\name{John}{Evans}

%\title{Resumé title}
% optional, remove / comment the line if not wanted

\address{1735 S Street NW}{Washington, DC  20009}{USA}
\phone[mobile]{+1~(617)~447~0030}                   
\email{john.g.evans.ne@gmail.com}                  
\social[twitter]{johnevans007}                    
\social[github]{quintusdias}                     

%-------------------------------------------------------------------------------
%            content
%-------------------------------------------------------------------------------
%            content
%-------------------------------------------------------------------------------

\end{document}


%% end of file `template.tex'.
