Dear Sirs,

I would like to place my name into consideration for the Python "wrangling" position.  Although most of my work experience has been more on the IT side of science institutions, I am a quick learner and have a lot (really, a lot)of experience wrangling data with Python, particular pandas.

In the past few years, Python has become my language of choice, and it was my primary tool working in GIS Dissemination at the National Weather Service.  This involved such tasks as analyzing Apache log files, writing monitoring scripts, running JMeter load tests, and more.  Pandas was probably the single most heavily used package for wrangling data (so, so many CSV variants...).  I also wrote monitoring scripts in Python for my two primary projects, nowCOAST and IDP-GIS.  The scripts had to run under the RHEL 6/7 system python, which meant python2.6/2.7, but I developed and tested them in both Python2 and Python3 in order to clearly learn the differences between the two standard libraries.

Since leaving NWS, I've recently been trying to bolster my data science / machine learning credentials through online courses at EdX, which has emphasized Python and Spark.  I also currently maintain a GitHub project called Glymur that provides a Python interface to the OpenJPEG library.  Glymur has been included in the Arch, Debian, Gentoo, and Ubuntu distributions of Linux, and MacPorts as well.  Working on Glymur allowed me to learn a great deal about other Python packages such as matplotlib, scikit-image, and maybe most important of all, the unittest module.  Learning how to write good tests with unittest has been an ongoing project, and I am constantly surprised at learning some new (or old) feature and wondering how I ever got along without it.

Before working for NOAA/NWS in various capacities, I was a developer at the MathWorks, where I authored a number of features as a member and eventual team lead of the Image and Scientific Data Formats team.  Working at the MathWorks was a good experience in software engineering practices such as continuous integration, refactoring, requirements gathering, functional design, design reviews including test plans, and long-range planning.

Thank you for your consideration.  Please do not hesitate to contact me if you require any further information.
