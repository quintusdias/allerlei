\documentclass[11pt]{moderncv}  % Sets the default text size to 11pt and class
                               % to article.
% moderncv themes
% style options are 'casual' (default), 'classic', 'oldstyle' and 'banking'
\moderncvstyle{casual}

% color options 'blue' (default), 'orange', 'green', 'red', 'purple',
% 'grey' and 'black'
\moderncvcolor{blue}

% to set the default font; use '\sfdefault' for the default sans serif font,
%   '\rmdefault' for the default roman one, or any tex font name
%\renewcommand{\familydefault}{\sfdefault}

% uncomment to suppress automatic page numbering for CVs longer than one page
%\nopagenumbers{}

% character encoding
\usepackage[utf8]{inputenc}

% adjust the page margins
\usepackage[scale=0.75]{geometry}

% if you want to change the width of the column with the dates
%\setlength{\hintscolumnwidth}{3cm}

% for the 'classic' style, if you want to force the width allocated to your
%   name and avoid line breaks. be careful though, the length is normally
%   calculated to avoid any overlap with your personal info; use this at your
%   own typographical risks...
%\setlength{\makecvtitlenamewidth}{10cm}
%% recipient data
\recipient
{Biocomplexity Institute}
{Charlottesville, VA}

% personal data
\name{John}{Evans}

%\title{Resumé title}
% optional, remove / comment the line if not wanted

\address{1735 S Street NW}{Washington, DC  20009}{USA}
\phone[mobile]{+1~(617)~447~0030}                   
\email{john.g.evans.ne@gmail.com}                  
\social[twitter]{johnevans007}                    
\social[github]{quintusdias}                     
%\extrainfo{additional information}             

%\photo[64pt][0.4pt]{picture}
% optional, remove / comment the line if not wanted.
% '64pt' is the height the picture must be resized to,
% 0.4pt is the thickness of the frame around it (put it to 0pt for no frame)
% and 'picture' is the name of the picture file

\hyphenation{MATLAB}

%-------------------------------------------------------------------------------
%            content
%-------------------------------------------------------------------------------
%            content
%-------------------------------------------------------------------------------
\begin{document}

\makecvtitle

%%-----       letter       ----------------------------------------------------
\opening{Dear sirs,}
\closing{Sincerely,}

% use an optional argument to use a string other than "Enclosure", or redefine
% \enclname
%\enclosure[Attached]{curriculum vit\ae{}}

\makelettertitle

I would like to place my name into consideration for the Software Engineer position at the Biocomplexity Institute.   I have a great deal of experience with programming and software support at scientific institutions as well as expertise in scientific data formats such as netCDF, HDF4 and HDF5, and feel that I would be well suited to the position.

My history with netCDF began with my first job out of college back in 1995, 
when I began working for the Atlantic Marine Geology branch of USGS in Woods Hole,
MA.  My boss was an oceanographer whose models were built upon netCDF (first ECOM-si then ROMS).  I built several tools to process netCDF input and output files using
MATLAB.  The netCDF interface to MATLAB was then a community-driven project.
In a much-later position at Rutgers, I took over the community-driven interface
and moved it to Sourceforge.  And finally, the position following saw me
coming on board the MathWorks on the Image and Scientific Data Formats team, where I transitioned netCDF support from the community model into the commercial product.  I also became proficient in HDF5 during the time of the relayering of netCDF4 on top of HDF5 during my time at the MathWorks, and I authored new features for HDF5 as well as improving old features.

In the past few years, however, Python has become my language of choice.  I currently
maintain a GitHub project called Glymur that provides a Python interface
to the OpenJPEG library.  Glymur has been included in the Arch, Debian,
Gentoo, and Ubuntu distributions of Linux, and MacPorts as well.
Working on Glymur allowed me to learn a great deal about other Python
packages such as matplotlib, scikit-image, and maybe most important of
all, the unittest module.  Learning how to write good tests with unittest
has been an ongoing project, and I am constantly surprised at learning
some new feature and wondering how I ever got along without it.

Bash and Python were my primary tool working in GIS Dissemination at the National
Weather Service.  This involved such tasks as analyzing Apache log
files, writing monitoring scripts to monitor both the ingest of weather data as well as the operational status of web mapping services and backend Postgresql databases, running JMeter load tests, and more.
Monitoring 24/7 was the primary focus of the position, however.  Since
leaving NWS, I've recently been trying to bolster my data science /
machine learning credentials through online courses at EdX, which has
emphasized Python and Spark.

Thank you for your consideration.  Please do not hesitate to contact
me if you require any further information.

\makeletterclosing
\end{document}


%% end of file `template.tex'.
